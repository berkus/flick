%% -*- mode: LaTeX -*-
%%
%% Copyright (c) 1997, 1999 The University of Utah and the Computer Systems
%% Laboratory at the University of Utah (CSL).
%%
%% This file is part of Flick, the Flexible IDL Compiler Kit.
%%
%% Flick is free software; you can redistribute it and/or modify it under the
%% terms of the GNU General Public License as published by the Free Software
%% Foundation; either version 2 of the License, or (at your option) any later
%% version.
%%
%% Flick is distributed in the hope that it will be useful, but WITHOUT ANY
%% WARRANTY; without even the implied warranty of MERCHANTABILITY or FITNESS
%% FOR A PARTICULAR PURPOSE.  See the GNU General Public License for more
%% details.
%%
%% You should have received a copy of the GNU General Public License along with
%% Flick; see the file COPYING.  If not, write to the Free Software Foundation,
%% 59 Temple Place #330, Boston, MA 02111, USA.
%%

%%%%%%%%%%%%%%%%%%%%%%%%%%%%%%%%%%%%%%%%%%%%%%%%%%%%%%%%%%%%%%%%%%%%%%%%%%%%%%%

This chapter illustrates the use of Flick for implementing a simple network
``phonebook.''  In our examples, a single server process will maintain one or
more phonebooks, with each phonebook containing a set of name-and-number
entries.  Client processes will be able to connect to the server in order to
add, remove, or locate phonebook entries.  The complete phonebook application
will be implemented in three different ways: first as a CORBA C program, then
as a CORBA C++ program, and finally as an ONC~RPC (C) program.  By reading
through these examples, you will learn how to use Flick to develop your own
network applications.


%%%%%%%%%%%%%%%%%%%%%%%%%%%%%%%%%%%%%%%%%%%%%%%%%%%%%%%%%%%%%%%%%%%%%%%%%%%%%%%

\section{The CORBA Phonebook in C}
\label{sec:The CORBA Phonebook in C}

%% -*- mode: LaTeX -*-
%%
%% Copyright (c) 1997, 1998, 1999 The University of Utah and
%% the Computer Systems Laboratory at the University of Utah (CSL).
%%
%% This file is part of Flick, the Flexible IDL Compiler Kit.
%%
%% Flick is free software; you can redistribute it and/or modify it under the
%% terms of the GNU General Public License as published by the Free Software
%% Foundation; either version 2 of the License, or (at your option) any later
%% version.
%%
%% Flick is distributed in the hope that it will be useful, but WITHOUT ANY
%% WARRANTY; without even the implied warranty of MERCHANTABILITY or FITNESS
%% FOR A PARTICULAR PURPOSE.  See the GNU General Public License for more
%% details.
%%
%% You should have received a copy of the GNU General Public License along with
%% Flick; see the file COPYING.  If not, write to the Free Software Foundation,
%% 59 Temple Place #330, Boston, MA 02111, USA.
%%

%%%%%%%%%%%%%%%%%%%%%%%%%%%%%%%%%%%%%%%%%%%%%%%%%%%%%%%%%%%%%%%%%%%%%%%%%%%%%%%

The source code for the CORBA phonebook application is contained in the
\filename{test/examples/phone/corba} directory of the Flick source tree.


%%%%%%%%%%%%%%%%%%%%%%%%%%%%%%%%%%%%%%%%%%%%%%%%%%%%%%%%%%%%%%%%%%%%%%%%%%%%%%%

\subsection{Interface Definition}
\label{subsec:CORBA:Interface Definition}

The CORBA IDL specification of our phonebook interface is straightforward:

\begin{verbatim}
   module data {
       typedef string<200> name;
       typedef string<20>  phone;

       struct entry { name n; phone p; };

       exception duplicate { phone p; };
       exception notfound  {};

       interface phonebook {
           void  add(in entry e)   raises (duplicate);
           void  remove(in name n) raises (notfound);
           phone find(in name n)   raises (notfound);
       };
   };
\end{verbatim}

Assuming that this interface is contained in a file \filename{phone.idl}, the
following commands will compile the phonebook specification into client stubs
and a server skeleton:\footnote{The actual commands issued by the
\filename{Makefile} are somewhat more complicated because they do not assume
that you have installed Flick on your system.  Read the \filename{Makefile} for
details.}

\begin{verbatim}
   flick-fe-newcorba phone.idl

   flick-c-pfe-corba -c -o phone-client.prc phone.aoi
   flick-c-pbe-iiop phone-client.prc
   # Final outputs: `phone-client.c' and `phone-client.h'.

   flick-c-pfe-corba -s -o phone-server.prc phone.aoi
   flick-c-pbe-iiop phone-server.prc
   # Final outputs: `phone-server.c' and `phone-server.h'.
\end{verbatim}


%%%%%%%%%%%%%%%%%%%%%%%%%%%%%%%%%%%%%%%%%%%%%%%%%%%%%%%%%%%%%%%%%%%%%%%%%%%%%%%

\subsection{Server Functions}
\label{subsec:CORBA:Server Functions}

For the server, Flick will create a \cfunction{main} function (found in
\filename{phone-server.c}).  To complete the server you must:

\begin{itemize}
  \item define the data structures that the server will need in order to manage
  its objects;

  \item write a \cfunction{register_objects} function that will instantiate the
  server's objects (as described in Section~\ref{subsec:Creating Object
  Implementations}); and

  \item write functions to implement the operations listed in the IDL file.
\end{itemize}

Our simple server will use a linked list to represent its collection of phone
book objects.  Each node in that list will be a `\ctype{struct impl_node_t}'
and will represent a single phonebook:

\begin{verbatim}
   typedef struct impl_node_t *impl_node;

   struct impl_node_t {
       CORBA_ReferenceData id;     /* The search key for this node. */
       data_phonebook obj;         /* CORBA Object ref for this.    */

       data_entry **pb;            /* The array of ptrs to entries. */
       int pb_elems;               /* # of entries in `pb'.         */
       int pb_size;                /* The size of the `pb' array.   */

       impl_node next;             /* Ptr to the next list element. */
   };

   /* `pb_impl_list' is the list of phonebook implementations. */
   impl_node pb_impl_list = 0;
\end{verbatim}

The \ctype{CORBA_ReferenceData} slot of each node will allow us to find the
data for a phonebook, given its CORBA object reference.\footnote{This example
uses a linked list to represent the server's set of phonebooks, but a more
sophisticated application would use a hash table, with keys generated by the
\cfunction{CORBA_Object_hash} function.}  The other slots should be obvious,
although you may be wondering where \ctype{data_entry} is defined.  It is
defined in the file \filename{phone-server.h}, which is created for you by
Flick.  The \ctype{data_entry} structure type is automatically generated from
the the \idl{entry} structure declaration in the source IDL file.

The next step is to write \cfunction{register_objects}, which will be called
from the Flick-generated \cfunction{main} function so that the server can
instantiate its objects and register them with the Flick runtime (the ORB).
The code for this function is shown below and is contained in the
\filename{phone-workfuncs.c} file in Flick's CORBA phonebook source directory.
This \cfunction{register_objects} function searches the command line for
options of the form ``\option{-I~\optionarg{name}}'' and creates one object for
each such option.  In the code below, note that
\cfunction{data_phonebook_server} is the name of the server skeleton function
that was generated for you by Flick.  That is the function that dispatches
requests to objects that support the \idl{phonebook} interface defined in our
IDL\@.

\begin{verbatim}
   void register_objects(CORBA_ORB o, CORBA_BOA b, int argc, char **argv,
                         CORBA_Environment *ev)
   {
       int i;

       orb = o; /* Globals: save ORB and BOA handles for later use. */
       boa = b;

       pb_impl_list = 0;

       for (i = 1; i < argc; i++) {
           if (!strcmp(argv[i], "-I") && (i + 1 < argc)) {
               CORBA_ReferenceData id;
               data_phonebook obj;
               impl_node list = pb_impl_list;

               /* Register object with name `argv[i + 1]'. */
               id._length = strlen(argv[i + 1]);
               id._maximum = id._length;
               id._buffer = (CORBA_char *)
                             CORBA_alloc(id._maximum * sizeof(CORBA_char));
               if (!id._buffer) {
                   signal_no_memory(ev);
                   return;
               }
               memcpy(id._buffer, argv[i + 1], id._length);

               obj = CORBA_BOA_create(boa, &id,
                                      "data::phonebook",
                                      &data_phonebook_server,
                                      ev);
               if (ev->_major != CORBA_NO_EXCEPTION)
                   return;

               /* Add a corresponding `impl_node' to our list. */
               pb_impl_list = (impl_node)
                              malloc(sizeof(*pb_impl_list));
               if (!pb_impl_list) {
                   signal_no_memory(ev);
                   return;
               }

               pb_impl_list->id = id;
               pb_impl_list->obj = obj;
               pb_impl_list->pb = 0;
               pb_impl_list->pb_elems = 0;
               pb_impl_list->pb_size = 0;
               pb_impl_list->next = list;
           }
       }

       /* Signal failure if no objects were registered. */
       if (!pb_impl_list)
           signal_initialize(ev);
   }
\end{verbatim}

We are now ready to write the server ``work functions'' --- the functions that
implement the operations defined in our phonebook interface.  Abbreviated code
for the \cfunction{data_phonebook_add} function (implementing the \idl{add}
operation defined in our IDL) is shown below.  Complete source code for all the
work functions is contained in the \filename{phone-workfuncs.c} file in the
example source code directory.

\begin{verbatim}
   void data_phonebook_add(data_phonebook obj, data_entry *arg,
                           CORBA_Environment *ev)
   {
       /* Find the object implementation --- see `find_impl' below. */
       impl_node impl = find_impl(obj, ev);
       int i;

       /* If we failed to find the implementation, return. */
       if (ev->_major != CORBA_NO_EXCEPTION)
           return;

       /* See if this entry is already in the phonebook. */
       for (i = 0; i < impl->pb_size; ++i) {
           if (impl->pb[i] && !strcmp(impl->pb[i]->n, arg->n)) {
               /* Found a duplicate!  Raise a `data_duplicate' exception. */
               data_duplicate *d =
                   (data_duplicate *) CORBA_alloc(sizeof(data_duplicate));

               if (!d) {
                   signal_no_memory(ev); return;
               }
               d->p = (CORBA_char *) CORBA_alloc(strlen(impl->pb[i]->p) + 1);
               if (!d->p) {
                   CORBA_free(d); signal_no_memory(ev); return;
               }
               strcpy(d->p, impl->pb[i]->p);
               CORBA_BOA_set_exception(boa, ev, CORBA_USER_EXCEPTION,
                                       ex_data_duplicate, d);
               return;
           }
       }

       /* Find an empty entry in `impl'; grow the phonebook if necessary. */
       i = find_empty_entry(impl, ev);
       if (ev->_major != CORBA_NO_EXCEPTION)
           return;

       /*
        * Allocate memory for the new entry.  Note that we have to copy the
        * `arg' data because CORBA says we can't keep pointers into `in' data
        * after this function has returned.
        */
       impl->pb[i] = (data_entry *) malloc(sizeof(data_entry));
       if (!impl->pb[i]) {
           signal_no_memory(ev); return;
       }

       impl->pb[i]->n = (char *) malloc(sizeof(char) * (strlen(arg->n) + 1));
       impl->pb[i]->p = (char *) malloc(sizeof(char) * (strlen(arg->p) + 1));
       if (!(impl->pb[i]->n) || !(impl->pb[i]->p)) {
           /* Free what we have allocated and signal an exception. */
           ... ; return;
       }

       /* Copy the `arg' information into our phonebook. */
       strcpy(impl->pb[i]->n, arg->n);
       strcpy(impl->pb[i]->p, arg->p);

       /* Increment the number of entries in our phonebook. */
       impl->pb_elems++;

       /* Success! */
       return;
   }
\end{verbatim}

The above function calls \cfunction{find_impl} to find the node that represents
a phonebook, given its CORBA object reference.  That helper function is shown
below.

\begin{verbatim}
   impl_node find_impl(data_phonebook obj, CORBA_Environment *ev)
   {
       impl_node this_impl = pb_impl_list;
       CORBA_ReferenceData *obj_id = CORBA_BOA_get_id(boa, obj, ev);

       if (ev->_major != CORBA_NO_EXCEPTION)
           return 0;

       while (this_impl) {
           if ((this_impl->id._length == obj_id->_length)
               && !memcmp(this_impl->id._buffer,
                          obj_id->_buffer,
                          obj_id->_length))
               break;
           this_impl = this_impl->next;
       }
       if (!this_impl) {
           /* This should never be reached. */
           signal_object_not_exist(ev);
       }

       CORBA_free(obj_id->_buffer);
       CORBA_free(obj_id);

       return this_impl;
   }
\end{verbatim}


%%%%%%%%%%%%%%%%%%%%%%%%%%%%%%%%%%%%%%%%%%%%%%%%%%%%%%%%%%%%%%%%%%%%%%%%%%%%%%%

\subsection{Client Program}
\label{subsec:Client Program}

For the phonebook client, Flick creates stubs that will send requests to the
server process and receive the server's replies.  You must write the client's
\cfunction{main} function.  Our client will be interactive, and its
\cfunction{main} function is shown below.  Note that our client uses
\cfunction{CORBA_ORB_string_to_object} to establish its connection to an object
that resides in the server.  Flick's IIOP runtime will cause the server to
print the names of its objects as they are registered.  The argument to our
client must be either the IOR-style or the URL-style name of an object in the
server; refer to Section~\ref{subsec:Creating Object References} for more
information about object names.

\begin{verbatim}
   int main(int argc, char **argv)
   {
       CORBA_ORB orb = 0;
       CORBA_Environment ev;
       data_phonebook obj;
       int sel, done;

       if (argc != 2) {
           fprintf(stderr, "Usage: %s <phone obj reference>\n", argv[0]);
           exit(1);
       }

       orb = CORBA_ORB_init(&argc, argv, 0, &ev);
       if (ev._major != CORBA_NO_EXCEPTION) {
           printf("Can't initialize the ORB.\n");
           exit(1);
       }

       obj = CORBA_ORB_string_to_object(orb, argv[1], &ev);
       if (ev._major != CORBA_NO_EXCEPTION) {
           printf("Can't convert `%s' into an object reference.\n",
                  argv[1]);
           exit(1);
       }

       done = 0;
       while (!done) {
           read_integer(("\n(1) Add an entry (2) Remove an entry "
                         "(3) Find a phone number (4) Exit: "),
                        &sel);
           switch(sel) {
           case 1:  add_entry(obj); break;
           case 2:  remove_entry(obj); break;
           case 3:  find_entry(obj); break;
           case 4:  done = 1; break;
           default: printf("Please enter 1, 2, 3, or 4.\n");
           }
       }
       return 0;
   }
\end{verbatim}

The client's \cfunction{add_entry} function invokes
\cfunction{data_phonebook_add}, which is the Flick-generated stub for the
\idl{add} operation.  It handles exceptions by calling the
\cfunction{handle_exception} function; the source for
\cfunction{handle_exception} can be found in the \filename{phonebook.c} file in
our example's source directory.

\begin{verbatim}
   void add_entry(data_phonebook obj)
   {
       data_entry e;
       char name[NAME_SIZE], phone[PHONE_SIZE];
       CORBA_Environment ev;

       e.n = name;
       e.p = phone;

       read_string("Enter the name: ", e.n, NAME_SIZE);
       read_string("Enter the phone number: ", e.p, PHONE_SIZE);

       data_phonebook_add(obj, &e, &ev);
       if (ev._major != CORBA_NO_EXCEPTION)
           handle_exception(&ev);
       else
           printf("`%s' has been added.\n", name);
   }
\end{verbatim}


%%%%%%%%%%%%%%%%%%%%%%%%%%%%%%%%%%%%%%%%%%%%%%%%%%%%%%%%%%%%%%%%%%%%%%%%%%%%%%%

\subsection{Compiling the Application}
\label{subsec:CORBA:Compiling the Application}

The \filename{test/examples/phone/corba} directory contains a simple
\filename{Makefile} for compiling the phonebook server and client programs.
You will need to edit the \filename{Makefile} slightly in order to suit your
build environment.  Once that is done, and you have built Flick and the IIOP
runtime, you should be able to type \program{make} to build the CORBA
phonebook.  Two programs will be created: \program{phoneserver}, the server,
and \program{phonebook}, the client.


%%%%%%%%%%%%%%%%%%%%%%%%%%%%%%%%%%%%%%%%%%%%%%%%%%%%%%%%%%%%%%%%%%%%%%%%%%%%%%%

\subsection{Using the Phonebook}
\label{subsec:CORBA:Using the Phonebook}

To run the application you must first start the phonebook server.  The server
expects to receive at least two arguments on the command line:

\begin{optionlist}
  \item[-OAport~\optionarg{portnumber}]
  %
  The port on which to run the server.

  \item[-I~\optionarg{name}]
  %
  The names of the object instances that the server will create.  This option
  may be repeated in order to create several object instances.  Each object
  must have a unique name.
\end{optionlist}

\noindent See Section~\ref{subsec:Server Command Line Options} for a list of
additional, optional arguments.  Once the \program{phoneserver} program is
running, you can invoke the \program{phonebook} program, giving it the IOR or
URL-style name of a server object.

\begin{verbatim}
   1 marker:~> phoneserver -OAport 1253 -I OfficeList -I DeptList
   Warning: no `-ORBipaddr' specified; using `marker.cs.utah.edu'.
   Object `OfficeList' is ready.
     URL:  iiop:1.0//marker.cs.utah.edu:1253/data::phonebook/OfficeList
     IOR:  IOR:0100000010000000646174613a3a70686f6e65626f6f6b0001000...
   Object `DeptList' is ready.
     URL:  iiop:1.0//marker.cs.utah.edu:1253/data::phonebook/DeptList
     IOR:  IOR:0100000010000000646174613a3a70686f6e65626f6f6b0001000...

   # Run a client on the same machine or on a different machine.

   1 fast:~> phonebook iiop:1.0//marker.cs.utah.edu:1253/...

   (1) Add an entry (2) Remove an entry (3) Find a phone number (4) Exit:
   ...
\end{verbatim}


%%%%%%%%%%%%%%%%%%%%%%%%%%%%%%%%%%%%%%%%%%%%%%%%%%%%%%%%%%%%%%%%%%%%%%%%%%%%%%%

%% End of file.




%%%%%%%%%%%%%%%%%%%%%%%%%%%%%%%%%%%%%%%%%%%%%%%%%%%%%%%%%%%%%%%%%%%%%%%%%%%%%%%

\section{The CORBA Phonebook in C++}
\label{sec:The CORBA Phonebook in C++}

%% -*- mode: LaTeX -*-
%%
%% Copyright (c) 1999 The University of Utah and the Computer Systems
%% Laboratory at the University of Utah (CSL).
%%
%% This file is part of Flick, the Flexible IDL Compiler Kit.
%%
%% Flick is free software; you can redistribute it and/or modify it under the
%% terms of the GNU General Public License as published by the Free Software
%% Foundation; either version 2 of the License, or (at your option) any later
%% version.
%%
%% Flick is distributed in the hope that it will be useful, but WITHOUT ANY
%% WARRANTY; without even the implied warranty of MERCHANTABILITY or FITNESS
%% FOR A PARTICULAR PURPOSE.  See the GNU General Public License for more
%% details.
%%
%% You should have received a copy of the GNU General Public License along with
%% Flick; see the file COPYING.  If not, write to the Free Software Foundation,
%% 59 Temple Place #330, Boston, MA 02111, USA.
%%

%%%%%%%%%%%%%%%%%%%%%%%%%%%%%%%%%%%%%%%%%%%%%%%%%%%%%%%%%%%%%%%%%%%%%%%%%%%%%%%

In this section we will show you how to make a CORBA C++ version of the
phonebook application.  The source code for this example can be found in the
\filename{test/examples/phone/tao-corbaxx} directory of the Flick distribution.

\emph{Note that you must acquire and install TAO version~\taoversion{}, the
real-time ORB from Washington University in St.\ Louis, before you will be able
to compile the CORBA C++ phonebook or any other Flick-generated CORBA C++ code.
See Section~\ref{sec:Building the Flick Runtime Libraries} for more
information.}


%%%%%%%%%%%%%%%%%%%%%%%%%%%%%%%%%%%%%%%%%%%%%%%%%%%%%%%%%%%%%%%%%%%%%%%%%%%%%%%

\subsection{Interface Definition}
\label{subsec:CORBAXX:Interface Definition}

The CORBA IDL specification of our phonebook, contained in the
\filename{phone.idl} file, is the same as the IDL shown previously in
Section~\ref{subsec:CORBA:Interface Definition}.  Because CORBA IDL is
independent of the target programming language, we can use the same IDL file to
generate both C and C++ versions of our phonebook application.
%
We produce our C++ stubs with the following commands:\footnote{The actual
commands issued by the \filename{Makefile} are somewhat more complicated
because they do not assume that you have installed Flick on your system.  Read
the \filename{Makefile} for details.}

\begin{verbatim}
   flick-fe-newcorba phone.idl

   flick-c-pfe-corbaxx -c -o phoneC.prc phone.aoi
   flick-c-pbe-iiopxx -f phoneS.h phoneC.prc
   # Final outputs: `phoneC.cc' and `phoneC.h'.

   flick-c-pfe-corbaxx -s -o phoneS.prc phone.aoi
   flick-c-pbe-iiopxx -F phoneC.h phoneS.prc
   # Final outputs: `phoneS.cc' and `phoneS.h'.
\end{verbatim}

\noindent The reasons for the \option{-f} and \option{-F} command line options
are described in Section~\ref{subsec:BE:Additional Comments}.
%% XXX --- We should \ref{subsubsec:BE:IIOP/C++ Back End Issues}, but doing so
%% confuses `TeX4ht'.  It makes a link with empty text, presumably because
%% subsubsections aren't displayed with numbers.


%%%%%%%%%%%%%%%%%%%%%%%%%%%%%%%%%%%%%%%%%%%%%%%%%%%%%%%%%%%%%%%%%%%%%%%%%%%%%%%

\subsection{Server Implementation}
\label{subsec:CORBAXX:Server Implementation}

Flick generates a server skeleton from the phonebook IDL\@.  To complete the
server, we need to write an actual implementation of the phonebook object
ourselves.  Writing an implementation class is just like writing a regular C++
class, but with some additional idioms imposed by CORBA\@.  Below is the
definition of our phonebook implementation class, called \ctype{phone_i}:

\begin{verbatim}
   class phone_i : public POA_data::phonebook
   {
   public:
       phone_i();
       ~phone_i();

       /* The functions that implement the `phone.idl' interface. */
       virtual void add(const data::entry &e, CORBA::Environment &);
       virtual void remove(const char *n, CORBA::Environment &);
       virtual data::phone find(const char *n, CORBA::Environment &);

   private:
       /* We implement the phonebook with a simple linked list. */
       struct impl_node_t {
           data::entry         entry;
           struct impl_node_t *next;
       };
       typedef struct impl_node_t *impl_node_ptr;

       /* The head of the linked list. */
       impl_node_ptr head;

       /*
        * `find_node' locates and returns an `impl_node_ptr' to an existing
        * node that contains the given name, or a null pointer if no node
        * contains the name.
        */
       impl_node_ptr find_node(const char *name);
   };
\end{verbatim}

The declaration of this class is made up of a constructor and destructor, some
virtual functions corresponding to the interface operations, and some private
members used in the implementation.

Notice that the \ctype{phone_i} class inherits from a special class called
\ctype{POA_data::phonebook}.  This \emph{POA class}, also called a
\emph{skeleton class}, is created for you by Flick and is the abstract base
class for all implementations of the \idl{data::phonebook} interface.  The
skeleton class takes care of marshaling and unmarshaling data for the methods
that implement the phonebook interface (\cfunction{add}, \cfunction{remove},
and \cfunction{find}).  Flick also generates special ``tie'' and ``collocated''
classes for each interface in the IDL; to learn more about these, consult the
CORBA specification and the TAO documentation.

Now that the \ctype{phone_i} class is declared, we need to write the methods
that implement the operations listed in our IDL\@.  Code for the
\cfunction{phone_i::add} method is shown below.  Complete source code for all
of the \ctype{phone_i} methods is contained in the \filename{phone_i.cc} file
in the example source code directory.

\begin{verbatim}
   void phone_i::add(const data::entry &e, CORBA::Environment &ACE_TRY_ENV)
   {
       impl_node_ptr node_ptr;

       /* Check for a duplicate name. */
       node_ptr = find_node(e.n);
       if (node_ptr) {
           /*
            * We already have this number.  Use the `ACE_THROW' macro to
            * throw an exception.
            */
           ACE_THROW(data::duplicate(node_ptr->entry.p));
       } else {
           /*
            * Make a new node and add it in.  Note that the structure
            * assignment below does a *deep* copy: the structure type is
            * defined in IDL, and the generated assignment operator does a
            * deep copy as required by the CORBA C++ language mapping.
            */
           node_ptr = new impl_node_t;
           node_ptr->entry = e; /* Deep copy. */
           node_ptr->next = head;
           head = node_ptr;
       }
   }
\end{verbatim}

The implementation of our phonebook operations is rather ordinary C++ code,
plus some special idioms for handling exceptions.
%
CORBA suggests that an IDL-to-C++ compiler should produce code that handles
CORBA-defined exceptions as regular C++ exceptions, whenever the target C++
compiler has working exception support.  Flick does not yet implement this code
style because many C++ compilers still have poor support for exceptions.
Instead, Flick supports the CORBA standard's \emph{alternate mapping} for
exceptions, which works with all C++ compilers.  In this mapping, CORBA
exceptions are communicated through an extra \ctype{CORBA::Environment}
parameter to the generated stubs, as in the standard CORBA C language mapping.
%
TAO provides some exception-related macros (e.g., \cfunction{ACE_THROW}) that
hide the details of exception handling, thus allowing our code to be largely
independent of the specific exception mapping implemented by Flick.  Refer to
the TAO documentation for more information about these facilities.

Once the \ctype{phone_i} implementation is complete, the only remaining task
for our server is to implement the \cfunction{main} function.  Our
\cfunction{main} function initializes the ORB and then creates a set of
phonebook objects as described by the server's command line.  (The command line
is parsed by the function \cfunction{parse_args}, not shown.)  When everything
is set up, we simply set the system in motion and let the server process
requests from clients.

\begin{verbatim}
   int main(int argc, char **argv)
   {
       TAO_ORB_Manager orb_manager;
       CORBA::Environment ACE_TRY_ENV;

       /* Use `try' because there might be an exception. */
       ACE_TRY {
           phone_impl *pi;

           /* Initialize the POA. */
           orb_manager.init_child_poa(argc, argv, "child_poa", ACE_TRY_ENV);
           ACE_TRY_CHECK;

           /* Parse the command line; creates `the_impls' list. */
           if (parse_args(argc, argv) != 0)
               return 1;

           /*
            * Walk through the list of implementations and attach each
            * one to the server.
            */
           for (pi = the_impls; pi; pi = pi->next) {
               /*
                * Add the head of the `pi' list to the POA and get the
                * IOR for the object.
                */
               CORBA::String_var ior
                   = (orb_manager.
                      activate_under_child_poa(pi->name,
                                               &pi->servant,
                                               ACE_TRY_ENV));
               ACE_TRY_CHECK;

               /* Give the IOR to the user. */
               ACE_DEBUG((LM_DEBUG, "Object `%s' is ready.\n", pi->name));
               ACE_DEBUG((LM_DEBUG, "  IOR:  %s\n", ior.in()));
           }
           /*
            * Make sure there is at least one object and then start the
            * server.
            */
           if (the_impls)
               orb_manager.run(ACE_TRY_ENV);
           else
               ACE_ERROR_RETURN((LM_ERROR,
                                 "No implementations were specified.\n"),
                                1);
           ACE_TRY_CHECK;
       }
       /* Catch any system exceptions. */
       ACE_CATCH (CORBA::SystemException, ex) {
           ex._tao_print_exception("System Exception");
           return 1;
       }
       ACE_ENDTRY;

       return 0;
   }
\end{verbatim}


%%%%%%%%%%%%%%%%%%%%%%%%%%%%%%%%%%%%%%%%%%%%%%%%%%%%%%%%%%%%%%%%%%%%%%%%%%%%%%%

\subsection{Client Program}
\label{subsec:CORBAXX:Client Program}

With the server side implemented, we can now create a client to allow for
interactions with the phonebook.  Flick creates the stubs that will send
requests to the server process and receive the server's replies.  A programmer,
however, must write the client program's \cfunction{main} function to
initialize the ORB and establish the client's connection to a phonebook, i.e.,
obtain a reference to a phonebook object that resides in the server.  The code
for our sample client \cfunction{main} function is shown below.

Our client establishes a connection by taking an IOR from the command line.
(The arguments ``\option{-k~\optionarg{IOR}}'' are parsed by the function
\cfunction{parse_args}, not shown.)  Once the client has established the
phonebook connection, we can treat the phonebook reference as a normal C++
object, which makes it easy to write the rest of the client.

The client is interactive so there is a small event loop that prompts the user
for requests.  When the user decides to exit, the \cfunction{main} function
terminates, and the connection to the server is closed.  (When we leave the
\cfunction{main} function, the `\ctype{_var}'-style object references are
automatically destroyed, with the result being that our connection to the
server is closed.)

\begin{verbatim}
   int main(int argc, char **argv)
   {
       CORBA::ORB_var orb;
       CORBA::Object_var object;
       data::phonebook_var pb;
       CORBA::Environment ACE_TRY_ENV;
       const char *op_name;
       int sel, done;

       /* Parse the command line; initializes `ior'. */
       if (parse_args(argc, argv) != 0)
           return 1;

       ACE_TRY {
           /* Initialize the ORB. */
           op_name = "CORBA::ORB_init";
           orb = CORBA::ORB_init(argc, argv, 0, ACE_TRY_ENV);
           ACE_TRY_CHECK;

           /* Get the object reference with the IOR. */
           op_name = "CORBA::ORB::string_to_object";
           object = orb->string_to_object(ior, ACE_TRY_ENV);
           ACE_TRY_CHECK;

           /* Narrow the object to a `data::phonebook'. */
           op_name = "data::phonebook::_narrow";
           pb = data::phonebook::_narrow(object.in(), ACE_TRY_ENV);
           ACE_TRY_CHECK;
       }
       ACE_CATCH (CORBA::Exception, ex) {
           ex._tao_print_exception(op_name);
           return 1;
       }
       ACE_ENDTRY;

       done = 0;
       while (!done) {
           read_integer(("\n(1) Add an entry (2) Remove an entry "
                         "(3) Find a phone number (4) Exit: "),
                        &sel);
           switch (sel) {
           case 1:  add_entry(pb); break;
           case 2:  remove_entry(pb); break;
           case 3:  find_entry(pb); break;
           case 4:  done = 1; break;
           default: printf("Please enter 1, 2, 3, or 4.\n");
           }
       }
       return 0;
   }
\end{verbatim}

The client's \cfunction{add_entry} function invokes the
\cfunction{data::phonebook::add} method, which is the Flick-generated stub for
the \idl{add} operation.  Because the stub can signal an exception, the
\cfunction{add_entry} function is careful to catch exceptions as shown below.
(As previously described in Section~\ref{subsec:CORBAXX:Server Implementation},
we use the \cfunction{ACE_*} macros provided by TAO to make our code more
portable across different C++ compilers and different ways of handling CORBA
exceptions.)

\begin{verbatim}
   void add_entry(data::phonebook_var obj)
   {
       data::entry e;
       char name[NAME_SIZE], phone[PHONE_SIZE];
       CORBA::Environment ACE_TRY_ENV;

       read_string("Enter the name: ", name, NAME_SIZE);
       read_string("Enter the phone number: ", phone, PHONE_SIZE);
       /*
        * Duplicate the strings; they will be freed when `e' is destroyed.
        */
       e.n = CORBA::string_dup(name);
       e.p = CORBA::string_dup(phone);

       ACE_TRY {
           obj->add(e, ACE_TRY_ENV);
           /* Check for exceptions. */
           ACE_TRY_CHECK;
           printf("`%s' has been added.\n", name);
           break;
       }
       ACE_CATCH(data::duplicate, ex) {
           /*
            * Catch a `data::duplicate' exception.  It contains the
            * phone number that is already in the database.
            */
           printf("A user exception was raised: ");
           printf("duplicate, phone = `%s'.\n", (const char *) ex.p);
       }
       ACE_CATCH(CORBA::Exception, ex) {
           /* Catch all other exceptions. */
           ex._tao_print_exception("data::phonebook::add");
       }
       ACE_ENDTRY;
   }
\end{verbatim}


%%%%%%%%%%%%%%%%%%%%%%%%%%%%%%%%%%%%%%%%%%%%%%%%%%%%%%%%%%%%%%%%%%%%%%%%%%%%%%%

\subsection{Compiling the Application}
\label{subsec:CORBAXX:Compiling the Application}

The \filename{test/examples/phone/tao-corbaxx} directory contains a
\filename{Makefile} for compiling the phonebook server and client programs.
You will need to edit the \filename{Makefile} slightly in order to suit your
build environment.  Once that is done, and you have built both Flick and TAO,
you should be able to type \program{make} to build the CORBA phonebook.  Two
programs will be created: \program{phoneserver}, the server, and
\program{phonebook}, the client.


%%%%%%%%%%%%%%%%%%%%%%%%%%%%%%%%%%%%%%%%%%%%%%%%%%%%%%%%%%%%%%%%%%%%%%%%%%%%%%%

\subsection{Using the Phonebook}
\label{subsec:CORBAXX:Using the Phonebook}

To run the application you must first start the phonebook server.  The server
expects to receive at least one argument on the command line:

\begin{optionlist}
  \item[-I~\optionarg{name}]
  %
  The names of the object instances that the server will create.  This option
  may be repeated in order to create several object instances.  Each object
  must have a unique name.
\end{optionlist}

Once the \program{phoneserver} program is running, you can invoke the
\program{phonebook} program, giving it the IOR of a server object.

\begin{verbatim}
   1 marker:~> phoneserver -I OfficeList -I DeptList
   Object `DeptList' is ready.
     IOR:  IOR:010000001700000049444c3a646174612f70686f6e65626f6f6b3...
   Object `OfficeList' is ready.
     IOR:  IOR:010000001700000049444c3a646174612f70686f6e65626f6f6b3...

   # Run a client on the same machine or on a different machine.

   1 fast:~> phonebook -k IOR:010000001700000049444c3a646174612f7068...

   (1) Add an entry (2) Remove an entry (3) Find a phone number (4) Exit:
   ...
\end{verbatim}


%%%%%%%%%%%%%%%%%%%%%%%%%%%%%%%%%%%%%%%%%%%%%%%%%%%%%%%%%%%%%%%%%%%%%%%%%%%%%%%

%% End of file.




%%%%%%%%%%%%%%%%%%%%%%%%%%%%%%%%%%%%%%%%%%%%%%%%%%%%%%%%%%%%%%%%%%%%%%%%%%%%%%%

\section{The ONC RPC Phonebook}
\label{sec:The ONC RPC Phonebook}

%% -*- mode: LaTeX -*-
%%
%% Copyright (c) 1997, 1999 The University of Utah and
%% the Computer Systems Laboratory at the University of Utah (CSL).
%%
%% This file is part of Flick, the Flexible IDL Compiler Kit.
%%
%% Flick is free software; you can redistribute it and/or modify it under the
%% terms of the GNU General Public License as published by the Free Software
%% Foundation; either version 2 of the License, or (at your option) any later
%% version.
%%
%% Flick is distributed in the hope that it will be useful, but WITHOUT ANY
%% WARRANTY; without even the implied warranty of MERCHANTABILITY or FITNESS
%% FOR A PARTICULAR PURPOSE.  See the GNU General Public License for more
%% details.
%%
%% You should have received a copy of the GNU General Public License along with
%% Flick; see the file COPYING.  If not, write to the Free Software Foundation,
%% 59 Temple Place #330, Boston, MA 02111, USA.
%%

%%%%%%%%%%%%%%%%%%%%%%%%%%%%%%%%%%%%%%%%%%%%%%%%%%%%%%%%%%%%%%%%%%%%%%%%%%%%%%%

In the following sections we will reimplement our phonebook application using
Flick's ONC~RPC tools rather than Flick's CORBA tools.  The source code for the
ONC~RPC phonebook is contained in the \filename{test/examples/phone/oncrpc}
directory of the Flick distribution.


%%%%%%%%%%%%%%%%%%%%%%%%%%%%%%%%%%%%%%%%%%%%%%%%%%%%%%%%%%%%%%%%%%%%%%%%%%%%%%%

\subsection{Interface Definition}
\label{subsec:ONCRPC:Interface Definition}

Again, the ONC~RPC IDL specification of the phonebook interface is
straightforward:

\begin{verbatim}
   typedef string name<200>;
   typedef string phone<20>;

   struct entry { name n; phone p; };

   program netphone {
       version firstphone {
           int     add(entry)   = 1;
           int     remove(name) = 2;
           phone   find(name)   = 3;
       } = 1;
   } = 58239;
\end{verbatim}

If you compare this interface to the one we previously defined using the CORBA
IDL (see Section~\ref{subsec:CORBA:Interface Definition}), you will notice some
minor differences.  Because ONC~RPC does not allow us to define explicit
exceptions, the return types of the \idl{add} and \idl{remove} operations have
changed in order to indicate success or failure.  Rather than change the return
type of the \idl{find} operation, however, our application will simply use an
empty phone number string to indicate search failures.

Assuming that the above interface is contained in a file \filename{phone.x},
the following commands will compile the phonebook specification into client
stubs and a server skeleton:\footnote{The actual commands issued by the
\filename{Makefile} are somewhat more complicated because they do not assume
that you have installed Flick on your system.  Read the \filename{Makefile} for
details.}

\begin{verbatim}
   flick-fe-sun phone.x

   flick-c-pfe-sun -c -o phone-client.prc phone.aoi
   flick-c-pbe-sun phone-client.prc
   # Final outputs: `phone-client.c' and `phone-client.h'.

   flick-c-pfe-sun -s -o phone-server.prc phone.aoi
   flick-c-pbe-sun phone-server.prc
   # Final outputs: `phone-server.c' and `phone-server.h'.
\end{verbatim}


%%%%%%%%%%%%%%%%%%%%%%%%%%%%%%%%%%%%%%%%%%%%%%%%%%%%%%%%%%%%%%%%%%%%%%%%%%%%%%%

\subsection{Server Functions}
\label{subsec:ONCRPC:Server Functions}

For the server, Flick will create a \cfunction{main} function (found in
\filename{phone-server.c}).  In order to complete the server, you must:

\begin{itemize}
  \item define the data structures that the server will need in order to manage
  its phonebook; and

  \item write functions to implement the operations listed in the IDL file.
\end{itemize}

In comparison with the CORBA version of our phonebook server, the ONC~RPC
server will be simpler because it will only manage one phonebook, not several.
(If we wanted our ONC~RPC server to manage multiple books, we would need to
change our IDL file so that we could pass a phonebook name to each operation.)
So, our ONC~RPC server simply needs to maintain a single array of phonebook
entries:

\begin{verbatim}
   entry **pb = 0;      /* The array of pointers to entries. */
   int pb_elems = 0;    /* # of entries in `pb'.             */
   int pb_size = 0;     /* The size of the `pb' array.       */
\end{verbatim}

The \ctype{entry} type definition is created by Flick from the \idl{entry}
definition in the original IDL file.  Now we are ready to write the server work
functions to implement the operations defined in our phonebook interface.  The
C prototypes of those functions are shown below:

\begin{list}{}{}
  \item{\cprototype{int *add_1(entry *arg, struct svc_req *obj);}}\\
  %
  Add an entry to the phonebook.  Return pointer to zero to indicate success,
  or pointer to non-zero to indicate an error.

  \item{\cprototype{int *remove_1(name *arg, struct svc_req *obj);}}\\
  %
  Remove the entry containing the given name from the phonebook.  Return
  pointer to zero to indicate success, or pointer to non-zero to indicate an
  error.

  \item{\cprototype{phone *find_1(name *arg, struct svc_req *obj);}}\\
  %
  Find the phone number corresponding to the given name in the phonebook.
  Return an empty string to indicate failure.
\end{list}

The name of each function is constructed from the name of the corresponding
operation defined in IDL, followed by an underscore and the version number of
the interface (in this case, 1).  The second argument to each function is a
``service request'' structure that would generally contain data about the
current request and the requesting client.  \emph{Flick's ONC/TCP runtime does
not currently initialize this structure!}  Flick-generated stubs include this
parameter simply to be prototype-compatible with \program{rpcgen}-generated
code.

Complete code for all of these functions is contained in the
\filename{phone-workfuncs.c} file in Flick's ONC~RPC phonebook source
directory.  Abbreviated code for \cfunction{add_1} is shown below.  Note that
as expected of ONC~RPC work functions, the return value is a pointer to a
statically allocated result.

\begin{verbatim}
   int *add_1(entry *arg, struct svc_req *obj)
   {
       static int result;
       int i;

       /* Return result: zero for success, or non-zero for error. */
       result = 1;

       /* See if this entry is already in the phonebook. */
       for (i = 0; i < pb_size; ++i) {
           if (pb[i] && !strcmp(pb[i]->n, arg->n)) {
               /* We found a duplicate!  Return an error code. */
               return &result;
           }
       }

       /* Find an empty entry in `pb'; grow the phonebook if necessary. */
       i = find_empty_entry();
       if (i == NULL_PB_INDEX)
           return &result;

       /*
        * Allocate memory for the new entry.  Note that we have to copy the
        * `arg' data because ONC RPC says we can't keep pointers into `in'
        * data after this function has returned.
        */
       pb[i] = (entry *) malloc(sizeof(entry));
       if (!pb[i])
           return &result;

       pb[i]->n = (char *) malloc(sizeof(char) * (strlen(arg->n) + 1));
       pb[i]->p = (char *) malloc(sizeof(char) * (strlen(arg->p) + 1));
       if (!(pb[i]->n) || !(pb[i]->p)) {
           /* Free what we have allocated and signal an exception. */
           ...
       }

       /* Copy the `arg' information into our phonebook. */
       strcpy(pb[i]->n, arg->n);
       strcpy(pb[i]->p, arg->p);

       /* Increment the number of entries in our phonebook. */
       pb_elems++;

       /* Success! */
       result = 0;
       return &result;
   }
\end{verbatim}


%%%%%%%%%%%%%%%%%%%%%%%%%%%%%%%%%%%%%%%%%%%%%%%%%%%%%%%%%%%%%%%%%%%%%%%%%%%%%%%

\subsection{Client Program}
\label{subsec:ONCRPC:Client Program}

You must provide the \cfunction{main} function for the client, which will allow
the user to invoke the Flick-generated stubs to communicate with the ONC~RPC
phonebook server.  The host name of the server is provided on the command line;
the values of \cidentifier{netphone} and \cidentifier{firstphone} are defined
in the file \filename{phone-client.h}, which is created by Flick.  The
\ctype{FLICK_SERVER_LOCATION} structure is described in
Section~\ref{subsec:Connecting to a Server}.

\begin{verbatim}
   int main(int argc, char **argv)
   {
       CLIENT client_struct, *c;
       FLICK_SERVER_LOCATION s;
       int sel, done;

       c = &client_struct;

       if (argc != 2) {
           fprintf(stderr, "Usage: %s <host>\n", argv[0]);
           exit(1);
       }

       s.server_name = argv[1];
       s.prog_num = netphone;
       s.vers_num = firstphone;
       create_client(c, s);

       done = 0;
       while (!done) {
           read_integer(("\n(1) Add an entry (2) Remove an entry "
                         "(3) Find a phone number (4) Exit: "),
                        &sel);
           switch(sel) {
           case 1:  add_entry(c); break;
           case 2:  remove_entry(c); break;
           case 3:  find_entry(c); break;
           case 4:  done = 1; break;
           default: printf("Please enter 1, 2, 3, or 4.\n");
           }
       }
       return 0;
   }
\end{verbatim}

The client's \cfunction{add_entry} function invokes \cfunction{add_1}, which is
the Flick-generated stub for the \idl{add} operation.  Exceptions are detected
by examining the return value of the stub function.  A null pointer indicates
an RPC failure; otherwise, \cidentifier{res} points to the return code for the
operation.

\begin{verbatim}
   void add_entry(CLIENT *c)
   {
       entry e;
       char name_array[NAME_SIZE], phone_array[PHONE_SIZE];
       int *result;

       e.n = name_array;
       e.p = phone_array;

       read_string("Enter the name: ", e.n, NAME_SIZE);
       read_string("Enter the phone number: ", e.p, PHONE_SIZE);

       result = add_1(&e, c);
       if (!result)
           printf("Error: bad RPC call for add_1.\n");
       else if (*result)
           printf("Error: `%s' not added, error code = %d.\n",
                  e.n, *result);
       else
           printf("`%s' has been added.\n", e.n);
   }
\end{verbatim}


%%%%%%%%%%%%%%%%%%%%%%%%%%%%%%%%%%%%%%%%%%%%%%%%%%%%%%%%%%%%%%%%%%%%%%%%%%%%%%%

\subsection{Compiling the Application}
\label{subsec:ONCRPC:Compiling the Application}

The \filename{test/examples/phone/oncrpc} directory contains a simple
\filename{Makefile} for compiling the phonebook server and client programs.
You will need to edit the \filename{Makefile} slightly in order to suit your
build environment.  Once that is done, and you have built Flick and the ONC/TCP
runtime, you should be able to type \program{make} to build the ONC~RPC
phonebook.  Two programs will be created: \program{phoneserver}, the server,
and \program{phonebook}, the client.


%%%%%%%%%%%%%%%%%%%%%%%%%%%%%%%%%%%%%%%%%%%%%%%%%%%%%%%%%%%%%%%%%%%%%%%%%%%%%%%

\subsection{Using the Phonebook}
\label{subsec:ONCRPC:Using the Phonebook}

To run the application you must first start the phonebook server.  Note that
unlike the CORBA phonebook server, our ONC~RPC server does not require any
command line arguments.  Once the \program{phoneserver} program is running, you
can invoke the \program{phonebook} program, giving it the name of the host on
which the server is running.

\begin{verbatim}
   1 marker:~> phoneserver
   Server sendbuf: 0 2 65536
   Server recvbuf: 0 2 65536

   # Run a client on the same machine or on a different machine.

   3 fast:~> phonebook marker
   Client sendbuf: 0 0 65536
   Client recvbuf: 0 0 65536

   (1) Add an entry (2) Remove an entry (3) Find a phone number (4) Exit:
   ...
\end{verbatim}


%%%%%%%%%%%%%%%%%%%%%%%%%%%%%%%%%%%%%%%%%%%%%%%%%%%%%%%%%%%%%%%%%%%%%%%%%%%%%%%

%% End of file.




%%%%%%%%%%%%%%%%%%%%%%%%%%%%%%%%%%%%%%%%%%%%%%%%%%%%%%%%%%%%%%%%%%%%%%%%%%%%%%%

%% End of file.

